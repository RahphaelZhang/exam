\usepackage[a4paper, hmargin=2.2cm, vmargin=2.0cm, marginparwidth=2cm, marginparsep=0.2cm]{geometry}

\usepackage{graphicx}
\usepackage{verbatim}
\usepackage{color}
\usepackage{multicol}
\usepackage{listings}
\usepackage{xcolor}
\usepackage[pdfauthor={Alexandru Ghitza},
pdftitle={Experimental Mathematics 2020}]{hyperref}
\hypersetup{colorlinks=true,linkcolor=blue,urlcolor=blue}
\urlstyle{rm}
\usepackage{float}
%\usepackage{sagetex}


\lstset{numbers=none,
  frame=leftline,
  framerule=1ex,
  framesep=1ex,
  rulecolor=\color{gray!30},
  xleftmargin=2ex,
  basicstyle={\ttfamily},
  columns=fixed,
  showstringspaces=false,
  commentstyle={\ttfamily\color{red}},
  keywordstyle={\ttfamily\color{blue}},
  stringstyle={\ttfamily},
  fontadjust=true,
  basewidth={1.09ex},
  breaklines=false,
}

% \usepackage{enumerate}
\usepackage{enumitem}
\setlist[enumerate,1]{label=(\alph*)}
\setlist[enumerate,2]{label=(\roman*)}
\usepackage{numprint}
\usepackage[short]{datetime}

\usepackage{epsf}
\usepackage{todonotes}

\usepackage[automark,markcase=ignoreuppercase]{scrlayer-scrpage}
\pagestyle{scrheadings}
\clearscrheadfoot
\lohead[]{MAST90053 ExpMath}
\rehead[]{\leftmark}
\cfoot[\pagemark]{\pagemark}
\renewcommand*{\headfont}{\bfseries\small}
% \setheadsepline{0.1pt}

\usepackage{dsfont}
\usepackage{booktabs}
\usepackage{amssymb}
\usepackage{amsmath}
\usepackage{amsthm}
\usepackage{amsfonts}
\usepackage{amscd}
\usepackage{MnSymbol}
\usepackage{epstopdf}
\usepackage[protrusion=true,expansion=true,verbose=true,final=true]{microtype}
\usepackage{datetime}
\usepackage{marginnote}
\usepackage{ifdraft}

\usepackage{tikz}
\usetikzlibrary{spy}
\usetikzlibrary{matrix,arrows,positioning}
\usetikzlibrary{decorations.markings}
\usetikzlibrary{decorations.fractals}
\usetikzlibrary{decorations.pathmorphing}
\usetikzlibrary{arrows.meta}
\usetikzlibrary{shapes.geometric}
\usetikzlibrary{calc}
%\tikzset{>={Stealth[scale=1,angle'=45]}}

\usepackage{tikz-cd}
\usepackage{alphalph}

\newcommand{\ZZ}{\mathbb{Z}}
\newcommand{\QQ}{\mathbb{Q}}
\newcommand{\FF}{\mathbb{F}}
\newcommand{\EE}{\mathbb{E}}
\newcommand{\CC}{\mathbb{C}}
\newcommand{\NN}{\mathbb{N}}
\newcommand{\RR}{\mathbb{R}}
\newcommand{\SSS}{\mathbb{S}}
\newcommand{\Aut}{\operatorname{Aut}}
\newcommand{\End}{\operatorname{End}}
\newcommand{\pr}{\operatorname{Pr}}
\newcommand{\jnf}{\operatorname{JNF}}
\newcommand{\sni}{\operatorname{SNI}}
\newcommand{\ctob}{{\mathcal{B}\leftarrow\mathcal{C}}}
\newcommand{\btoc}{{\mathcal{C}\leftarrow\mathcal{B}}}
\newcommand{\im}{\operatorname{im}}
\newcommand{\id}{\operatorname{id}}
\newcommand{\tr}{\operatorname{tr}}
\newcommand{\rank}{\operatorname{rank}}
\newcommand{\isom}{\operatorname{isom}}
\newcommand{\Fix}{\operatorname{Fix}}
\newcommand{\lcm}{\operatorname{lcm}}
\newcommand{\longto}{\longrightarrow}
\newcommand{\SL}{\operatorname{SL}}
\newcommand{\PSL}{\operatorname{PSL}}
\newcommand{\PSp}{\operatorname{PSp}}
\newcommand{\Sp}{\operatorname{Sp}}
\newcommand{\PSU}{\operatorname{PSU}}
\newcommand{\SU}{\operatorname{SU}}
\newcommand{\U}{\operatorname{U}}
\newcommand{\SO}{\operatorname{SO}}
\newcommand{\so}{\Rightarrow}
\newcommand{\GL}{\operatorname{GL}}
\newcommand{\Stab}{\operatorname{Stab}}
\newcommand{\Span}{\operatorname{Span}}
\newcommand{\proj}{\operatorname{proj}}
\newcommand{\actson}{\rcirclearrowright}
\renewcommand{\Re}{\operatorname{Re}}
\renewcommand{\O}{\operatorname{O}}
\newcommand{\oh}{\mathcal{O}}
\newcommand{\one}{\mathds{1}}
\newcommand{\gen}{\Span}
\newcommand{\R}{\RR}
\renewcommand{\Im}{\operatorname{Im}}
\newcommand{\bb}{\mathbf{b}}
\newcommand{\cc}{\mathbf{c}}
\newcommand{\arccot}{\operatorname{arccot}}

\newtheorem{thm}{Theorem}
\newtheorem{lem}[thm]{Lemma}
\newtheorem{cor}[thm]{Corollary}
\newtheorem{prop}[thm]{Proposition}
\newtheorem*{wop}{Well-Ordering Property}
\newtheorem*{pmi}{Principle of Mathematical Induction}
\newtheorem*{ftar}{Fundamental Theorem of Arithmetic}
\newtheorem*{ftal}{Fundamental Theorem of Algebra}
\theoremstyle{definition}
\newtheorem{exm}[thm]{Example}
\newtheorem{rem}[thm]{Remark}
\newtheorem{exe}[thm]{Exercise}
\newtheorem{ques}[thm]{Question}
\newtheorem*{sol}{Exercise}


\newcommand{\mnote}[1]{%
  \marginnote{\small #1}
}

\newcommand{\ses}[7]{%
\begin{tikzpicture}[scale=1] %\scriptsize
  \matrix(m)[matrix of math nodes,
             row sep=2.6em,
             column sep=2em,
             text height=1.5ex,
             text depth=0.25ex,
             ampersand replacement=\&]
             {{#1}\&{#2}\&{#4}\&{#6}\&{#7}\\};
  \path[->,font=\scriptsize,>=angle 90]
  (m-1-1) edge (m-1-2)
  (m-1-2) edge node[auto] {$#3$} (m-1-3)
  (m-1-3) edge node[auto] {$#5$} (m-1-4)
  (m-1-4) edge (m-1-5);
\end{tikzpicture}
}

\newcommand{\mapdef}[5]{%
\begin{tikzpicture}[scale=1]
  \matrix(m)[matrix of math nodes,
             row sep=0.1em,
             column sep=2em,
             text height=1.5ex,
             text depth=0.25ex,
             ampersand replacement=\&]
             {{#1}\&{#2}\&{#3}\\{}\&{#4}\&{#5}\\};
  \path[->,>=angle 90] (m-1-2) edge (m-1-3);
  \path[|->,>=angle 90] (m-2-2) edge (m-2-3);
\end{tikzpicture}
}

\newcommand{\col}[1]{\begin{bmatrix} #1 \end{bmatrix}}

\tikzset{
  mirror scope/.is family,
  mirror scope/angle/.store in=\mirrorangle,
  mirror scope/center/.store in=\mirrorcenter,
  mirror setup/.code={\tikzset{mirror scope/.cd,#1}},
  mirror scope/.style={mirror setup={#1},spy scope={
      rectangle,lens={rotate=\mirrorangle,yscale=-1,rotate=-1*\mirrorangle},size=80cm}},
}
\newcommand\mirror[1][]{\spy[overlay,#1] on (\mirrorcenter) in node at (\mirrorcenter)}

\tikzset{
  rotate scope/.is family,
  rotate scope/angle/.store in=\rotateangle,
  rotate scope/center/.store in=\rotatecenter,
  rotate setup/.code={\tikzset{rotate scope/.cd,#1}},
  rotate scope/.style={rotate setup={#1},spy scope={
      rectangle,lens={rotate=\rotateangle},size=80cm}},
}
\newcommand\rotate[1][]{\spy[overlay,#1] on (\rotatecenter) in node at (\rotatecenter)}

\newcommand{\tdo}[1]{\ifdraft{\todo{#1}}{}}
\newcommand{\ds}{\displaystyle}


\newmuskip\pFqskip
\pFqskip=6mu
\mathchardef\pFcomma=\mathcode`, % keep a copy of the comma

\newcommand*\pFq[5]{%
  \begingroup
  \begingroup\lccode`~=`,
    \lowercase{\endgroup\def~}{\pFcomma\mkern\pFqskip}%
  \mathcode`,=\string"8000
  {}_{#1}F_{#2}\biggl[\genfrac..{0pt}{}{#3}{#4};#5\biggr]%
  \endgroup
}


\renewcommand{\ge}{\geqslant}
\renewcommand{\geq}{\geqslant}
\renewcommand{\le}{\leqslant}
\renewcommand{\leq}{\leqslant}
